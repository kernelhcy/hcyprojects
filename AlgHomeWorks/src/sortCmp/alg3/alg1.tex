\documentclass[a4paper, 11pt]{article}
\usepackage{amsmath, amssymb, amsthm}
\usepackage{xeCJK}
\usepackage{fontspec}
\usepackage{xunicode}
\usepackage{xltxtra}

\setCJKfamilyfont{tt}{SimSun}
\setmainfont{Times New Roman} 		%默认字体,默认英文字体。
\setCJKmainfont{SimSun} 		%中文默认字体
\setCJKmonofont{Times New Roman}
\CJKsetecglue{} 			%中英间隔

%\XeTeXlinebreaklocale "zh"  % 表示用中文的断行
%\XeTeXlinebreakskip = 0pt plus 1pt % 多一点调整的空间设置字体。

\newcommand{\cn}{\fontspec{Courier New}}
%字体大小
\newcommand{\ltwo}{\fontsize{18pt}{27pt}\selectfont} 		%小二
\newcommand{\three}{\fontsize{16pt}{24pt}\selectfont}		%三号
\newcommand{\lthree}{\fontsize{15pt}{22.5pt}\selectfont}	%小三
\newcommand{\four}{\fontsize{14pt}{21pt}\selectfont}		%四号
\newcommand{\lfour}{\fontsize{13pt}{18pt}\selectfont}		%小四
\newcommand{\five}{\fontsize{10.5pt}{15.75pt}\selectfont}	%五号
\newcommand{\lfive}{\fontsize{9pt}{13.5pt}\selectfont}		%小五

\usepackage{listings}
\lstset{language=Java}%这条命令可以让LaTeX排版时将C++键字突出显示
\lstset{breaklines}%这条命令可以让LaTeX自动将长的代码行换行排版
%\lstset{extendedchars=false}%这一条命令可以解决代码跨页时,章节标题,页眉等汉字不显示的问题
\lstset{xleftmargin=28pt}
\lstset{tabsize=4}
\lstset{columns=flexible}
\lstset{basicstyle=\cn\lfive}

%%%%%%%%%%%%%%%%%%%%%%%%
%%%%%%%%%%%正文%%%%%%%%%%
%%%%%%%%%%%%%%%%%%%%%%%%
\begin{document}

\title{{\Huge 算法分析与设计第三次作业\\}}
\author{黄丛宇 2010212439}
\date{\today}

\maketitle

\section{实验环境}
\begin{itemize}
	\item CPU: Intel(R) Core(TM)2 Duo CPU T5870 2.00GHz
	\item MEM: 1GB
	\item OS : Debian 5.0 (1GB swap)
	\item Java: java version "1.6.0\_21"
\end{itemize}
\section{排序算法比较}

由于Java的Random类之内返回$0 \sim 2 ^ {31}$范围内的随机数,因此,可以使用下面的方法获得$0 \sim 2 ^ {32}$的随机数。

通过Java的Random类的nextInt函数得到两个$0 \sim 2 ^ {31}$范围的随机数,取这两个数的低16位,拼接成一个$0 \sim 2 ^ {32}$范围内的随机数。
\begin{lstlisting}
	/**
	 * 使用两个随机的int数的低16位拼接成一个32位的随机数。
	 * @return
	 */
	private int getNextUint()
	{
		int re = 0;
		int leftPart, rightPart;
		leftPart = random.nextInt();
		rightPart = random.nextInt();
		re = leftPart;
		re <<= 16;
		rightPart &= 0xffff;	//取rightPart的低16位
		re += rightPart;
		return re;
	}
\end{lstlisting}

通过上面的函数得到的是$0 \sim 2 ^ {32}$范围内的无符号随机数,Java中没有无符号数,因此只能把这些数当做有符号数来比较。本实验中通过下面的函数实现无符号数的比较。

将一个数看作是十六进制的,从高到低一次比较其十六进制对应的位的值,得到大小关系。代码如下:
\begin{lstlisting}
	private static boolean uless(int a, int b)
	{
		int aa, bb;
		for(int i = 28; i >= 0; i -= 4){
			/*
			 * 将a和b看作是十六进制的数,aa和bb中存储的是
			 * a和b的每一位的数值。
			 * 如,a为0x3df32ad3,那么aa中存放的就分别是
			 * 3,d,f,3,2,a,d,3
			 */
			aa = (a >> i) & 0xf;
			bb = (b >> i) & 0xf;
			if(aa < bb){
				return true;
			}
			else if(aa > bb){
				return false;
			}
		}
		//a==b
		return false;
	}
\end{lstlisting}

\section{课后习题}
\subsection{习题7.3 Stooge sort}
a.Ans:

首先,当元素的个数小于等于三个的时候,算法可以正确的对其进行排序。

当元素个数大于三个的时候。在算法中,每次将数组A分成三等分,分别表述为X,Y,Z。其中X为[i, i + k], Y为(i + k, j - k),Z为[j - k, j],由于k取的是(j-i)/3的下底,所以,Y的长度会大于等于X和Z的长度。

在第6行的递归调用中,算法将X和Y部分的数据排成有序的。如果其中的某一个元素的最终位置在Z中,那么,这个元素此时只可能在Y中。假如这个元素在X中,那么Y中的所有元素都比这个元素大,那么也就是Y中的所有元素的最终位置都在Z中,但是,这样就造成在最终位置在Z中的元素个数大于Z的长度,因此假设不成立。

此时,最终位置在Z中的元素只存在与Y和Z中。在第7行的递归调用中,可以使Z中的所有元素都在其最终位置上。那么,第8行的递归调用将使X和Y中的元素都在其最终位置上,最终,所有的元素都在其最终位置上。

因此,此算法可以将数组A的元素正确排序。

b.Ans:

由算法可得递归式:
\begin{displaymath}
	T(n) = 3T(2n/3)+\Theta(1)
\end{displaymath}
由主定理可得:
\begin{displaymath}
	T(n) = \Theta(n ^ {log_{1.5}{3}}) = \Theta{n^{2.71}}
\end{displaymath}

c.Ans:

插入排序的时间复杂度是$\Theta(n^2)$,归并排序,堆排序和快速排序的时间复杂度都是$\Theta(nlgn)$。都要低于这个算法的$\Theta(n^{2.71})$,所以这几个教授浪得虚名。

\subsection{习题8.3-4}

对所有的元素开根号,然后乘以100,取下底。不同的值得到的结果也不同。这时候元素的范围在0 $\sim$ 100n之间,利用Counting sort可以在O(n)的时间内完成排序。

\subsection{习题8.4-4}

使用桶排序。对于桶i,存放距离原点的距离在下面的范围内:
\begin{displaymath}
	(\frac{\sqrt{i - 1}}{\sqrt{n\pi}}, \frac{\sqrt{i}}{\sqrt{n\pi}}]
\end{displaymath}

第一个桶在:
\begin{displaymath}
	[0, \frac{\sqrt{1}}{\sqrt{n\pi}}]
\end{displaymath}

这个环的面积是1/n。由于n个点在单位圆上出现的概率是一样的,因此,这个环中点个数的期望值是1。因此这个桶排序的时间复杂度的期望值就是$\Theta(n)$
\end{document}
